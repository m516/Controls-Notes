\boldentry{Variables and Symbols}





\entry{$x$}{
\begin{itemize}
    \item state vector (state space) (\href{https://en.wikipedia.org/wiki/State_space}{W}) \\
        Type: $ \mathbb{R}^n $
\end{itemize} 
}





\entry{$H$}{
\begin{itemize}
    \item Hamiltonian matrix 
    \item Hamiltonian (Hamiltonian mechanics) (\href{https://en.wikipedia.org/wiki/State_space}{W}) \\
        type: $ \mathbb{R}^n \rightarrow \mathbb{R} $
        \begin{itemize}
            \item Assuming discrete time linear system: 
            \[H_k = L(x_k, u_k, k) + p_{k+1}^{T} f(x_k, u_k, k)\]
        \end{itemize} 
            
            
    
\end{itemize} 
}





\entry{$\mathcal{L}$}{
\begin{itemize}
    \item the Lagrangian \\
    type: $ \mathbb{R}^n \rightarrow \mathbb{R} $
\end{itemize} 
}





\entry{$B$}{
\begin{itemize}
    \item A ball, defined as 
    \[B\left(x_0,\epsilon\right) = \left\{x\in\mathbb{R}^n : \norm{x-x_0} \leq \epsilon\right\}\]
\end{itemize} 
}





\entry{$C$}{
\begin{itemize}
    \item $C^1$ = Continuously differentiable, i.e. the first derivative is continuous.
    \item $C^n$ = The $n^{\text{th}}$ derivative is continuous.
    \item $\mathbb{C}$: the set of all complex numbers $a+bi$ where $a$ and $b$ are real and $i=\sqrt{-1}$
\end{itemize} 
}








\entry{$e_\#$}{
\begin{itemize}
    \item The \#th unit vector\\
        \[\qquad e_1 = \begin{pmatrix} 1 \\ 0 \\ 0 \\\vdots \\ 0 \end{pmatrix},
          \qquad e_2 = \begin{pmatrix} 0 \\ 1 \\ 0 \\\vdots \\ 0 \end{pmatrix},
          \qquad e_3 = \begin{pmatrix} 0 \\ 0 \\ 1 \\\vdots \\ 0 \end{pmatrix},
          \qquad \hdots \]
\end{itemize} 
}






\entry{$\nabla$}{
The del operator, which represents one of many long but similar operators on a vector field $v \in \mathbb{R}^n$.
\begin{itemize}
    \item $\nabla f$: Gradient of a function $f : \mathbb{R}^n \rightarrow \mathbb{R}$, 
        returning an $n$-dimensional vector. (\href{https://en.wikipedia.org/wiki/Del}{W}) \\
        This vector points in the direction of the greatest increase, and its magnitude is the slope. \\
        
        For example, a mountain climber could approximate the shape of a convex 
        mountain as a function $f_{mountain}$ that computes the altitude given 
        some latitude and longitude (assuming a very small mountain very far 
        from the poles). In other words, 
        $f_{mountain} : \mathbb{R}^2 \rightarrow \mathbb{R}$. The climber could 
        know which direction to climb to summit the peak: it's the direction 
        $\nabla f$, and the grade or slope of the mountain is $\left| \nabla f \right|$\\

        Note that if $n = 1$, $\nabla f$ is the standard derivative of $f$. \\
        Formally speaking:
        \[\nabla f = \displaystyle\sum_{i=1} ^{n} \frac{\partial f}{\partial x_i} e_i\]
    \item $\nabla \cdot \Vec{v}$: The divergence of a vector field $\Vec{v}$
    \item $\nabla \times \Vec{v}$: The curl of a vector field $\Vec{v}$
    \item $\Delta f$: the Laplace operator on a function $f : \mathbb{R}^n \rightarrow \mathbb{R}$,
        equivalent to the divergence of the gradient of $f$, i.e. 
        \[\delta f = \nabla^2 f = \nabla \cdot \nabla f\]
\end{itemize} 
}



 

\entry{$J$}{
\begin{itemize}
    \item Cost to go function \\
    type: $ \mathbb{R}^n \rightarrow \mathbb{R} $
\end{itemize} 
}



\entry{$p$}{
\begin{itemize}
    \item Lagrange multiplier (\href{https://en.wikipedia.org/wiki/Lagrange_multiplier}{W})
\end{itemize} 
}



\entry{$\left<\text{expr}\right>$}{
\begin{itemize}
    \item Lie bracket notation (\href{https://en.wikipedia.org/wiki/Lie_bracket_of_vector_fields}{W}) \\
            $\left<a, b\right> = b^T a$

    
\end{itemize} 
}




\entry{$\left\Vert\text{expr}\right\Vert$}{
\begin{itemize}
    \item Vector norm (TODO)
    \item Matrix norm (TODO)
    \item Functional norm (TODO)
    \item \textbf{Norm of a system $y(u,t)=h(u(t))$}, where $y$ is an $n$-dimensional the output of the system, $u$ is an $m$-dimensional control vector (TODO)
    
\end{itemize} 
}