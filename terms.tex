
\boldentry{Terms}




\entry{  
    Lipschitz Continuity \\
    \href{https://en.wikipedia.org/wiki/Lipschitz_continuity}{W},
    \href{https://math.berkeley.edu/~mgu/MA128ASpring2017/MA128ALectureWeek9.pdf}{UC Berkley}
}{
    Lipschitz continuous functions are continuous and differentiable almost anywhere in a domain. \\\\
    Given a domain $D$ and a function $f: D \rightarrow \mathbb{R}, D \in \mathbb{R}^n$, \\
    $f$ is Lipschitz continuous if $\exists L>0$ such that $|f(x) - f(y)| < L ||(x-y)|| \forall x,y \in D$
}



\entry{  
    Hessian \\
    \href{https://en.wikipedia.org/wiki/Hessian_matrix}{W},
    \href{https://www.khanacademy.org/math/multivariable-calculus/applications-of-multivariable-derivatives/quadratic-approximations/a/the-hessian}{Kahn Academy},
    \href{https://mathworld.wolfram.com/Hessian.html}{Wolfram}
}{
\begin{itemize}
    \item A $2n$ x $2n$ matrix of all 2nd order partial derivatives of some function $f : \mathbb{R}^n \rightarrow \mathbb{R}$
    \item The determinant of a Hessian matrix
\end{itemize} 
}



\entry{  
    definite \\
    \href{https://en.wikipedia.org/wiki/Positive-definite_function}{W}
}{
\emph{Warning: this definition does not appear to be common outside of controls}\\

Given a real-valued, continuously differentiable function $V(x) : \mathbb{R} \rightarrow \mathbb{R}$ \\
$V(x)$ can be classified as
\begin{itemize}
    \item 
        \textbf{(globally) positive semidefinite} if 
        \[
            V(x) \geq 0 \qquad \forall x \in \mathbb{R}
        \]
        \begin{center}
            \textit{($v$ is greater than or equal to 0 regardless of $x$)}
        \end{center}
    
    \item 
        \textbf{(globally) positive definite} if positive semidefinite AND
        \[
            V(x) = 0 \iff x = 0
        \]
        \begin{center}
            \textit{($V(x)$ is zero if and only if $x$ is zero)}
        \end{center}
        
    \item 
        \textbf{(globally) negative semidefinite} if 
        \[
            V(x) \leq 0 \qquad \forall x \in \mathbb{R}
        \]
        \begin{center}
            \textit{($v$ is less than or equal to 0 regardless of $x$)}
        \end{center}
    
    \item 
        \textbf{(globally) negative definite} if negative semidefinite AND
        \[
            V(x) = 0 \iff x = 0
        \]
        \begin{center}
            \textit{($V(x)$ is zero if and only if $x$ is zero)}
        \end{center}

        
    \item 
        \textbf{locally positive definite (l.p.d)} if 
        \[
            V(x) \geq 0 \qquad \forall x \in N
        \]
        where $N$ is a small open neighborhood containing $\Vec{0}$
        \begin{center}
            \textit{($v$ is greater than or equal to 0 regardless of $x$ in some small open neighborhood $N$ that contains the zero vector)}
        \end{center}
        \begin{center}
            \textbf{AND}
        \end{center}
        \[
            V(x) = 0 \iff x = 0
        \]
        \begin{center}
            \textit{($V(x)$ is zero if and only if $x$ is zero)}
        \end{center}

        Note that the criteria for a function to be locally positive definite are similar, but more relaxed than, those for globally positive definite functions.

        
    \item 
        \textbf{positive definite on some domain $D \in \mathbb{R}^n$} if \\
        we only care if the conditions for positive definite functions hold for all $x$ in $D$.
    
    
\end{itemize} 
} % end definite






\entry{  
    Stability
}{
\begin{itemize}
    \item (Lyapunov) stability (TODO)
    \item Asymptotic stability  (TODO)
    \item Exponential stability (TODO)
    \item Uniform stability (TODO)
    \item Global stability (TODO)
    \item L-stability (TODO)
    \item I/O L-stability (TODO)
    \item Small-signal I/O L-stability (TODO)
    \item Small-signal finite-gain L-stability (TODO)
    
\end{itemize} 
}






\entry{  
    Class K function
}{
\begin{itemize}
    \item (TODO)
\end{itemize} 
}






\entry{  
    Radially Unbounded function
}{
\begin{itemize}
    \item (TODO)
\end{itemize} 
}






\entry{  
    $\sup$ (supremum)
}{
    Like a maximum of a functions, but includes limits that aren't necessarily a part of the domain of the function.
    (TODO)
}





\entry{  
    Hurwitz
}{
\begin{itemize}
    \item \textbf{Hurwitz (polynomial)}: \\
        A polynomial whose roots that are all in the left-half plane. (In other words, the real part of every root is strictly negative)
    \item \textbf{Hurwitz (matrix)} (\href{https://en.wikipedia.org/wiki/Hurwitz_matrix}{W}): \\
        A square matrix whose characteristic polynomial is Hurwitz, meaning all eigenvalues are in the left-half plane. (In other words, the real part of every eigenvalue is strictly negative)
    \item \textbf{Routh-Hurwitz stability criterion} (\href{https://ieeexplore.ieee.org/document/165530}{IEEE}): \\
        TODO
\end{itemize} 

Any hyperbolic fixed point (or equilibrium point) of a continuous dynamical system is locally asymptotically stable if and only if the Jacobian of the dynamical system is Hurwitz stable at the fixed point.

A system is stable if its control matrix is a Hurwitz matrix. 

The negative real components of the eigenvalues of the matrix represent negative feedback. Similarly, a system is inherently unstable if any of the eigenvalues have positive real components, representing positive feedback. 
}





\entry{  
    Zero-state observable
}{
    A time-invariant system of the form
    \[ \begin{cases} 
          \dot{x} = f(x, u) \\
          y = h(x, u)
       \end{cases}
    \]
    
    is zero-state observable if
    
    \[ \begin{cases} 
          y \equiv 0 \\
          u \equiv 0
    \end{cases} \implies x \equiv 0
    \]

    In other words, when $u=0$, any nonzero state behavior will be observed at the output ($y\neq0$)
}
